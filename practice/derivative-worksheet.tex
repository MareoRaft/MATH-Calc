\documentclass[reqno,psamsfonts]{amsart}


%-------Packages---------
\usepackage[margin=.5in]{geometry}
\usepackage{nopageno}
\usepackage{amssymb,amsfonts}
\usepackage[all,arc]{xy}
\usepackage{enumerate}
\usepackage{mathrsfs}
\usepackage{tikz}
\usepackage{tikz-cd}
\usepackage{graphicx}
\usepackage{tkz-euclide}

\DeclareMathOperator{\Hom}{Hom}

%--------Theorem Environments--------
%theoremstyle{plain} --- default
\newtheorem{thm}{Theorem}[section]
\newtheorem{cor}[thm]{Corollary}
\newtheorem{prop}[thm]{Proposition}
\newtheorem{lem}[thm]{Lemma}
\newtheorem{conj}[thm]{Conjecture}
\newtheorem{quest}[thm]{Question}

\theoremstyle{definition}
\newtheorem{defn}[thm]{Definition}
\newtheorem{defns}[thm]{Definitions}
\newtheorem{con}[thm]{Construction}
\newtheorem{exmp}[thm]{Example}
\newtheorem{exmps}[thm]{Examples}
\newtheorem{notn}[thm]{Notation}
\newtheorem{notns}[thm]{Notations}
\newtheorem{addm}[thm]{Addendum}
\newtheorem{exer}[thm]{Exercise}

\theoremstyle{remark}
\newtheorem{rem}[thm]{Remark}
\newtheorem{rems}[thm]{Remarks}
\newtheorem{warn}[thm]{Warning}
\newtheorem{sch}[thm]{Scholium}

\makeatletter
\let\c@equation\c@thm
\makeatother
\numberwithin{equation}{section}

\bibliographystyle{plain}

%--------Meta Data: Fill in your info------
\title{Math 1210\\Derivative Worksheet}

\begin{document}
\maketitle
\thispagestyle{empty}
\noindent

\subsection*{Limit Definition of the Derivative}
Let $f$ be a function and $x$ be a real number. Then the derivative of $f$ at $x$, denoted $f'(x)$ or $\frac{d}{dx}[f(x)]$, is defined by
\begin{align*}
f'(x)=\lim\limits_{h\to 0}\frac{f(x+h)-f(x)}{h}\hspace{2em}\text{provided this limit exists.}
\end{align*}
If the above limit exists, we say that $f$ is differentiable at $x$, otherwise we say that $f$ is not differentiable at $x$.

\subsection*{Geometric Interpretation of the Derivative} If $f$ is differentiable at $x$, then $f'(x)$ is the slope of the line tangent to the graph of $f$ at the point $(x, f(x))$.

\subsection*{Rate of Change Interpretation of the Derivative} If $f$ is differentiable at $x$, then $f'(x)$ is the instantaneous rate of change of $f$ with respect to $x$.

\subsubsection*{Examples}
\begin{enumerate}
\item If $p(t)$ is position at time $t$, then $p'(t)$ is instantaneous velocity at time $t$.
\item If $v(t)$ is velocity at time $t$, then $v'(t)$ is instantaneous acceleration at time $t$.
\item If $C(x)$ is the cost of producing $x$ goods, then $C'(x)$ is marginal cost.
\end{enumerate}

\subsection*{Differentiation Rules}
\begin{enumerate}
\item Let $f(x)= c$ be a constant function. Then, $f'(x) = 0$.
\item Let $f$ be differentiable at $x$ and let $c$ be a constant. Then
\begin{align*}
\frac{d}{dx}[cf(x)]= c\frac{d}{dx}[f(x)]
\end{align*}

\item (Power Rule) Let $n$ be any real number. Then,
\begin{align*}
\frac{d}{dx}x^n=nx^{n-1}
\end{align*}

\item (Sum Rule) Let $f$ and $g$ be two functions such that $f$ and $g$ are both differentiable at $x$. Then,
\begin{align*}
(f+g)'(x) = f'(x)+g'(x)
\end{align*}

\item (Product Rule)
Let $f$ and $g$ be two functions such that $f$ and $g$ are both differentiable at $x$. Then
\begin{align*}
(fg)'(x) = f'(x)g(x)+f(x)g'(x)
\end{align*}

\item (Quotient Rule) Let $f$ and $g$ be two functions such that $g(x)\neq 0$ and $f$ and $g$ are both differentiable at $x$. Then,

\begin{align*}
\left(\frac{f}{g}\right )'(x) = \frac{f'(x)g(x)-f(x)g'(x)}{[g(x)]^2}
\end{align*}

\item (Chain Rule) Let $f$ and $g$ be two functions such that $f$ is differentiable at $x$ and $g$ is differentiable at $f(x)$. Then,
\begin{align*}
(g\circ f)'(x) = g'(f(x))f'(x)
\end{align*}

\item (Generalized Power rule)
Let $f$ be differentiable at $x$ and let $n$ be a real number. Then
\begin{align*}
\frac{d}{dx}[f(x)]^n=n[f(x)]^{n-1}f'(x)
\end{align*}
\end{enumerate}

\begin{thm}
Let $f$ be a function. If $f$ is differentiable at $x$, then $f$ is continuous at $x$.
\end{thm}
An equivalent way to say the above theorem is that if $f$ is not continuous at $x$, then $f$ is not differentiable at $x$. This is the contrapositive of the theorem 0.1.

\subsection*{Warning:} The converse of theorem 0.1 is not true. In other words, $f$ being continuous at $x$ \textbf{DOES NOT} imply that $f$ is differentiable at $x$. Here is an example illustrating that:

\subsubsection*{Example} Let $f(x) = |x|$. Then $f$ is continuous at $0$, but not differentiable at $0$.

\newpage
\subsection*{Practice Problems}

\begin{enumerate}
\vspace{1em}
\item In each of the following problems calculate $f'(x)$.
\vspace{1em}
\begin{enumerate}
\item $f(x) = (x+1)^7+x^2+3215$
\vspace{3em}
\item $f(x) = (x^2+\pi)\sqrt{3x^2+8}$
\vspace{3em}
\item $f(x) = (-6x^8+(2x+1)^{4/3}+1)^{2/7}$
\vspace{3em}
\item $f(x) = \dfrac{x^2}{\sqrt{7x^4+x^2+1}}$
\vspace{3em}
\item $f(x) = \dfrac{(20-x^7+x)^2}{1-x}$
\vspace{3em}
\end{enumerate}

\item Suppose $f(4)=-27$ and $f'(4)=9$. Let $h(x) = f(x^3+x-6)$. Calculate $h'(2)$.
\vspace{3em}
\item Find an equation of the line tangent to $f(x) =3x^{7/3}+x$ at $x=8$.
\vspace{3em}
\item Let $f(x) = 11\sqrt{x}$. Use the limit definition of the derivative to calculate $f'(x)$.
\vspace{3em}
\item (Challenging) Let $g(x) = x^{2/3}+4$. Use the limit definition of the derivative to calculate $g'(x)$.  (Note: this problem is difficult, and would not appear on an exam)
\vspace{3em}
\item Let $h(x) = x^{3}$. Use the limit definition of the derivative to calculate $h'(x)$.
\end{enumerate}
\end{document}

