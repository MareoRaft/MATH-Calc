\documentclass[12pt, landscape]{article}
%\usepackage{../notesheets}
\usepackage[margin=1cm, bottom=3cm, top=2cm]{geometry}
\usepackage{textpos}
\usepackage{amssymb}
\usepackage{amsthm} % http://ctan.org/pkg/ams\begin{thrm}
\usepackage{hyperref}
\pagenumbering{gobble}
\newtheoremstyle{mainstyle}
  {20pt plus 4pt minus 4pt} % Space above
  {70pt plus 4pt minus 4pt} % Space below
  {} % Body font
  {} % Indent amount
  {\bfseries} % Theorem head font
  {.} % Punctuation after theorem head
  {.5em} % Space after theorem head
  {} % Theorem head spec (can be left empty, meaning `normal')
\newtheoremstyle{longstyle}
  {20pt plus 4pt minus 4pt} % Space above
  {170pt plus 4pt minus 4pt} % Space below
  {} % Body font
  {} % Indent amount
  {\bfseries} % Theorem head font
  {.} % Punctuation after theorem head
  {.5em} % Space after theorem head
  {} % Theorem head spec (can be left empty, meaning `normal')

\theoremstyle{mainstyle} % this applies to ALL following new theorems % SEE research seminar, week 10 writeup for a possible way to exclude amsthm and [thm].
\newtheorem{thrm}[thm]{Theorem}
\newtheorem{defi}[thm]{Definition}
\newtheorem{coro}[thm]{Corollary}
\newtheorem{propo}[thm]{Proposition}
\newtheorem{lemm}[thm]{Lemma}
\newtheorem{examp}[thm]{Example}

\theoremstyle{longstyle}
\newtheorem{ex}[thm]{Challenge}



\everymath{\displaystyle}

\newcommand{\vs}{\vspace{8pt}}



%%%%%%%%%%%%%%%%%%%%%%%%%%%%%%%%%%%%%%%%%%%%%%%%%%
\author{Math 1210}
\title{Keeping Special Points and Properties of Functions Straight}
\date{}

\begin{document}
\maketitle
\begin{tabular}{c|c|c|c}
  Type
  &Form
  &Relations
  &Example Sentence \\
  \hline
  Critical Point
  &\((a,f(a))\)
  &\parbox{9cm}{\((a,f(a))\) is on the graph of
    \(f\), so \(a\) is in the domain of
    \(f\). \\
  \(f'(a) = 0\), \(f'(a)\) does not exist, or...\\
  \(a\) is called a critical number and \(f(a)\) is called a critical value.}
  &\parbox{5cm}{The critical points of \(f(x) = x^2-x+2\) are
    \((0,2)\) and \((1,2)\).} \\
  \hline
  Increasing/decreasing function
  & \parbox{4cm}{\(f(x)\) is increasing/decreasing on \((a,b)\).}
  & \parbox{9cm}{If \(f(x)\) is differentiable on \((a,b)\), then
    \begin{itemize}
    \item \(f'(x) > 0\) on \((a,b) \implies f\) is increasing on \((a,b)\).
    \item \(f'(x) < 0\) on \((a,b) \implies f\) is decreasing on \((a,b)\).
    \item \(f'(x) = 0\) on \((a,b) \implies f\) is constant on \((a,b)\).
    \end{itemize}
    }
  & \parbox{5cm}{\(f(x) = 3x^4 - 4x^3 - 12x^2 + 5\) is increasing on
    \((-\infty, 0)\) and \((2,\infty)\) and decreasing on \((0,2)\).} \\
  \hline
  Relative extrema
  & \parbox{4cm}{A function has a relative extrema at \(x=c\). The
    relative extrema } 
\end{tabular}
\end{document}