\documentclass[12pt, a4paper]{article}
\usepackage{../notesheets}

%%%%%%%%%%%%%%%%%%%%%%%%%%%%%%%%%%%%%%%%%%%%%%%%%%
\author{Math 1210}
\title{Notesheet. Section 1.4}
\date{}

\begin{document}
\maketitle
\nameline
%%%%%%%%%%%%%%%%%%%%%%%%%%%%%%%%%%%%%%%%%%%%%%%%%%
\begin{defi}
  A \de{linear equation} is
\end{defi}
\begin{defi}
  The \de{slope} of a line is
\end{defi}
\begin{defi}
  The \de{\(y\)-intercept} of a line is
\end{defi}
\begin{ex}
  Consider a straight line going through points \((-1,-2)\) and
  \((1,4)\). What is the slope of this line? Can you formulate an
  equation for this line?
\end{ex}
\begin{ex}
  A taxi company charges you \(\$4.50\) for getting in a taxi and
  then \(\$1.75\) for every mile you ride. Represent your
  total cost as a linear function of miles. How much will a 4 mile
  ride cost?
\end{ex}
\begin{ex}
  The relationship between temperature in degrees Farenheit and degree
  Celsius is linear. Given that \(32^\circ F = 0^\circ C\) and
  \(212^\circ F = 100^\circ C\), write a linear function for
  degrees Farenheit given degrees Celsius.
\end{ex}
\begin{defi}
  A line is \de{parallel} to another line when
\end{defi}
\begin{defi}
  A line is \de{perpendicular} to another line when
\end{defi}
\begin{ex}
  Let \(L\) be the line defined by \(y = 3x+4\). Give an equation for a
  line that is parallel to \(L\) and passes through \((2,7)\). Give an
  equation for a line that is perpendicular to \(L\) and passes
  through the point \((2,7)\). Plot these lines. Finally, if \(f(x) = 3x+4\), what is
  \(f^{-1}(x)\)?
\end{ex}
%%%%%%%%%%%%%%%%%%%%%%%%%%%%%%%%%%%%%%%%%%%%%%%%%%
\end{document}
