\documentclass[12pt, a4paper]{article}
\usepackage{../notesheets}
\usepackage{epsdice}
\usepackage{tikz}
\usetikzlibrary{shapes.geometric, arrows}

\newcommand{\Var}{\operatorname{Var}}
%%%%%%%%%%%%%%%%%%%%%%%%%%%%%%%%%%%%%%%%%%%%%%%%%%
\author{Math 1220}
\title{Notesheet. Section 9.1: Differential Equations} 
\date{}

\begin{document}
\maketitle
\nameline
%%%%%%%%%%%%%%%%%%%%%%%%%%%%%%%%%%%%%%%%%%%%%%%%%%
\begin{defi}
  \begin{itemize}
  \item A \de{differential equation} is an equation involving
    \vspace{0.5in}
  \item A \de{solution} of a differential equation is
    \vspace{0.5in}
  \item A \de{general solution} of a differential equations is
    \vspace{0.5in}
  \item A \de{particular solution} is 
  \end{itemize}
\end{defi}
\begin{ex}
  Consider the differential equation \(y+y'' = 0\) (from the ``trig
  identity homework'').
  \begin{enumerate}
  \item What are \(3\) solutions to this differential equation?
    \vspace{1in}
  \item What is the general solution to this differential equation?
  \end{enumerate}
\end{ex}
\vspace{-1in}
\begin{defi}
  An \de{initial value problem (IVP)} is problem that requires solving
  a differential equation with some ``initial conditions.''
\end{defi}
\begin{ex}
  \begin{enumerate}
  \item   Given that \(y=Ce^{-x}+x-1\) is a general solution of \(y'+y=x\),
  find the particular solution of the IVP \[
    y'+y=x \text{ and } y(0) = 2
  \]
  \vspace{1in}
  \item Given that \(y=C_1 x^3 + C_2x^2\) is a general solution of
    \(x^2y''-4xy'+6y = 0\), find the particular solution of the IVP \[
      x^2y''-4xy'+6y = 0, y(2) = 0, y'(2) = 4
    \]
  \end{enumerate}
\end{ex}
\vspace{-1in}
\begin{thrm}
  Let \(Q(t)\) be the amount of some quantity at time \(t\). Recall
  tha the rate of change of \(Q\) is given by \(\frac{dQ}{dt}\). Then,
  \begin{enumerate}
  \item \(Q\) increases (decreases) at a fixed rate \(k \iff
    \frac{dQ}{dt} = \)
  \item \(Q\) grows (decays) at a rate proportional to some quantity \(A \iff
    \frac{dQ}{dt} = \)
  \item \(Q\) grows (decays) at a rate jointly proportional to \(A\) and \(B
    \iff \frac{dQ}{dt} =\)
  \end{enumerate}
\end{thrm}
\vspace{-1in}
\begin{ex}
  Write out (but do not solve) an initial value problem for each of
  the following situations.
  \begin{enumerate}
  \item The world population at the beginning of \(2018\) was \(7.6\)
    billion. Setup a model for the world population over time assuming
    that the 
    population will continue to grow at a 
    rate of approximately \(2\%\) per year.
    \vspace{0.25in}
  \item A radioactive substance decays at a rate directly proportional
    to the current mass of the substance. Setup a model for the mass
    of the substace over time assuming that you start with \(10\)
    grams.
    \vspace{0.25in}
  \item During a flu epidemic, \(5\%\) of \(300\) Math 1220 students
    have contracted influenza at time \(t=0\). The rate at which they
    contract influenza is jointly proportional to the number of
    students who already have influenze and the number of students who
    have yet to be infected. \(20\%\) of students have contracted the
    flu by the \(10\)th day. 
  \end{enumerate}
\end{ex}
%%%%%%%%%%%%%%%%%%%%%%%%%%%%%%%%%%%%%%%%%%%%%%%%%% 
\end{document}
