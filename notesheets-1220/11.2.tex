\documentclass[12pt, a4paper]{article}
\usepackage{../notesheets}
\usepackage{epsdice}

\newcommand{\Var}{\operatorname{Var}}
%%%%%%%%%%%%%%%%%%%%%%%%%%%%%%%%%%%%%%%%%%%%%%%%%%
\author{Math 1220}
\title{Notesheet. Section 11.2: Infinite Sequences} 
\date{}

\begin{document}
\maketitle
\nameline
%%%%%%%%%%%%%%%%%%%%%%%%%%%%%%%%%%%%%%%%%%%%%%%%%%
\begin{defi}
  An \de{infinite sequence} \(\{a_n\}\) is a function whose domain is
  \\

  The \de{terms} of the sequence are\\
  
  So, the \de{\(n\)th term} is 
\end{defi}
\begin{rmk}
  Sometimes \(\{a_n\}\) is denoted \(\{a_n\}_{n=1}^\infty\). A
  sequence also can begin at any natural number \(k\), e.g.
  \begin{enumerate}
  \item \(\left\{ \frac{n}{n+1} \right\}_{n=1}^\infty = \)
  \item \(\left\{ (-1)^n n \right\}_{n=2}^\infty =\)
  \item \(\left\{ (2n+1)! \right\}_{n=0}^\infty =\)
  \end{enumerate}
\end{rmk}
\begin{ex}
  Find a formula for the \(n\)th term of
  \begin{enumerate}
  \item \(\{a_n\}_{n=1}^\infty = \left\{\frac{1}{2}, \frac{1}{4},
      \frac{1}{6}, \frac{1}{8}, \frac{1}{10}, \ldots \right\}\)
  \item \(\{a_n\}_{n=1}^\infty = \left\{ -\frac{4}{5}, \frac{8}{9},
      -\frac{16}{11}, \frac{32}{14}, -\frac{60}{17}, \ldots\right\}\)
  \end{enumerate}
\end{ex}
\vspace{-0.5in}
\begin{defi}
  The sequence \(\{a_n\}_{n=k}^\infty\) is called \de{convergent} if
  \\

  The sequence is called \de{divergent} if 
\end{defi}
\begin{thrm}
  The ``limit laws''  hold for sequences as well: Assume \[
    \lim_{n \to \infty} a_n = A < \infty \text{ and } \lim_{n \to
      \infty} b_n = B < \infty
  \]
  \begin{enumerate}
  \item For \(c\) a constant, \(\lim_{n \to \infty} ca_n = \)
  \item \(\lim_{n \to \infty} (a_n \pm b_n) = \)
  \item \(\lim_{n \to \infty} a_n b_n = \)
  \item \(\lim_{n \to \infty} \frac{a_n}{b_n} = \)
  \end{enumerate}
\end{thrm}
\vspace{-1in}
\begin{ex}
  Determine if the following sequences converge or diverge. If the
  converge, give the limit.
  \begin{enumerate}
  \item \(\{a_n\} = \left\{ \left(\frac{1}{2}\right)^n \right\}\)
  \item \(\{b_n\} = \left\{ \frac{2n^2+n}{3n^2+1} \right\}\)
  \item \(\{c_n\} = \left\{ \frac{2n^2+n}{3n^3+1} \right\}\)
  \item \(\{d_n\} = \left\{ \frac{2n^2+n}{3\sqrt{n}+1} \right\}\)
  \end{enumerate}
\end{ex}
\vspace{-2in}
\begin{thrm}
  The rate of growth of \(n! \gg e^n \gg n^k\). In other words
  \begin{itemize}
  \item \(\lim_{n \to \infty} \frac{e^n}{n!} = 0 = \lim_{n \to \infty}
    \frac{n^k}{n!} = \lim_{n \to \infty} \frac{n^k}{e^n}\)
  \item \(\lim_{n \to \infty} \frac{n!}{e^n}, \lim_{n \to \infty}
    \frac{n!}{n^k},\) and \(\lim_{n \to \infty} \frac{e^n}{n^k}\) all
    do not exist and the corresponding sequences tend towards infinity.
  \end{itemize}
\end{thrm}
\vspace{-1in}
\begin{ex}
  Of the microprocessors manufactured by a microelectronics firm for
  use in regularing fuel consumption in automobiles, \(1.5\%\) are
  defective. It can be shown that the probability of getting at least
  one defective microprocessor in a random sample of \(n\)
  microprocessors is \(f(n) = 1-(0.985)^n\). Consider the sequence
  \(\{a_n\}\) defined by \(a_n = f(n)\). What is \(\lim_{n \to \infty}
  a_n\) and interpret the result.
\end{ex}
\vspace{-1in}
%%%%%%%%%%%%%%%%%%%%%%%%%%%%%%%%%%%%%%%%%%%%%%%%%% 
\end{document}
