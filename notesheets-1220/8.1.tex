\documentclass[12pt, a4paper]{article}
\usepackage{../notesheets}
%%%%%%%%%%%%%%%%%%%%%%%%%%%%%%%%%%%%%%%%%%%%%%%%%%
\author{Math 1220}
\title{Notesheet. Section 8.1: Functions of Several Variables}
\date{}

\begin{document}
\maketitle
\nameline
%%%%%%%%%%%%%%%%%%%%%%%%%%%%%%%%%%%%%%%%%%%%%%%%%%
\vspace{-0.5in}
\begin{defi}
  A real-valued \de{function of two variables} \(f\) consists of
  \begin{enumerate}
  \item A set \(A\) of
    \vspace{0.3in}
  \item A rule that associates with each ordered pair in the domain of
    \(f\) 
  \end{enumerate}
\end{defi}
\vspace{-0.5in}
\begin{ex}
  If a principal of \(P\) dollars is deposited in an account earning
  interest at the rate of \(r\)/year compounded continuously, then the
  accumulated amount at the end of \(t\) years is given by \[
    A = f(P,r,t) = P e^{rt} \text{ dollars}
  \]
  Find the accumulated amount at the end of \(10\) years if a sum of
  \(\$10,000\) is deposited in an account earning interest at the rate
  of \(10\%\)/year.
\end{ex}
\vspace{-1in}
\begin{ex}
  What is the domain of \(f(x,y) = xy\)? What about \(f(x,y) =
  \frac{1}{xy}\)? Finally, what about \(f(x,y) =
 \ln(y+1) \cdot \sqrt{x-1}\)? Sketch these domains as regions in the \(xy\)-plane.
\end{ex}
\vspace{0.5in}
\begin{defi}
  The \de{three-dimensional Cartesian coordinate system} is \\
  \vspace{0.3in}\\
  The \de{graph} of a function of two variables is all points of the
  form 
\end{defi}
\begin{defi}
  Given a function \(f(x,y)\) in two variables, if \(c\) is some value
  of \(f\), then the \de{trace} of the graph of \(f\) in the plane
  \(z=c\) is \\
  \vspace{0.3in}\\
  Furthermore, a \de{level curve} is 
\end{defi}
\vspace{1in}
\begin{ex}
  Sketch the contour map of \(f(x,y) = x+y\). What is the domain and
  range of this function? Find the level curve thta contains the point \((3,4)\).
\end{ex}
\begin{ex}
  Sketch the contour map of \(f(x,y) = x^2+y^2\). What is the domain
  and range of this function? 
\end{ex}
%%%%%%%%%%%%%%%%%%%%%%%%%%%%%%%%%%%%%%%%%%%%%%%%%% 
\end{document}
