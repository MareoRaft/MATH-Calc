\documentclass[12pt, a4paper]{article}
\usepackage{../notesheets}
\usepackage{epsdice}
\usepackage{tikz}
\usetikzlibrary{shapes.geometric, arrows}

\newcommand{\Var}{\operatorname{Var}}
%%%%%%%%%%%%%%%%%%%%%%%%%%%%%%%%%%%%%%%%%%%%%%%%%%
\author{Math 1220}
\title{Notesheet. Section 11.5: Power Series and Taylor Series} 
\date{}

\begin{document}
\maketitle
\nameline
%%%%%%%%%%%%%%%%%%%%%%%%%%%%%%%%%%%%%%%%%%%%%%%%%%
\begin{rmk}
  Recall that, for a differential function \(f(x)\), we can
  \emph{approximate} \(f(x)\) near \(x=a\) with the \emph{linear equation} \[
    f(x) \approx f(a) + f'(a)(x-a)
  \]
  We want to take this idea further.
\end{rmk}
\begin{defn}
  A \de{power series} centered at \(x=a\) is a series of the form
\end{defn}
\vspace{0.5in}
\begin{ex}
  Are the following series power series?
  \begin{enumerate}
  \item \(\sum_{n=0}^\infty x^n\)
  \item \(\sum_{n=0}^\infty x^{n-1}\)
  \item \(\sum_{n=0}^\infty x^2(x-1)^n\)
  \end{enumerate}
\end{ex}
\vspace{-2in}
\begin{ex}
  When is \(\sum_{n=0}^\infty x^n\) convergent and when is it divergent?
\end{ex}
\pagebreak
\begin{defn}
  \begin{enumerate}
  \item The \de{interval of convergence} (IoC) for a power series
    \vspace{0.75in}
  \item The \de{radius of convergence} (RoC) is defined to be
    \vspace{0.5in}
  \end{enumerate}
\end{defn}
\begin{thrm}
  The radius of convergence for \(\sum_{n=0}^\infty a_n(x-a)^n\) is
  given by \[
    R = 
  \]
\end{thrm}
\begin{ex}
  Find the radius of convergence and the interval of convergence for
  the following power series:
  \begin{enumerate}
  \item \(\sum_{n=0}^\infty x^n\)
    \vspace{0.5in}
  \item \(\sum_{n=0}^\infty n! (x-1)^n\)
    \vspace{0.5in}
  \item \(\sum_{n=0}^\infty \frac{n^3(x+1)^n}{(n+1)!}\)
    \vspace{0.5in}
  \item \(\sum_{n=0}^\infty (2x+6)^n\)
  \end{enumerate}
\end{ex}
\vspace{-1.5in}
%%%%%%%%%%%%%%%%%%%%%%%%%%%%%%%%%%%%%%%%%%%%%%%%%% 
\end{document}
