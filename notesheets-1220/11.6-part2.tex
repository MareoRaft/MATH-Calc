\documentclass[12pt, a4paper]{article}
\usepackage{../notesheets}
\usepackage{epsdice}
\usepackage{tikz}
\usetikzlibrary{shapes.geometric, arrows}

\newcommand{\Var}{\operatorname{Var}}
%%%%%%%%%%%%%%%%%%%%%%%%%%%%%%%%%%%%%%%%%%%%%%%%%%
\author{Math 1220}
\title{Notesheet. Section 11.6 Part II: More
  Taylor Series} 
\date{}

\begin{document}
\maketitle
\nameline
%%%%%%%%%%%%%%%%%%%%%%%%%%%%%%%%%%%%%%%%%%%%%%%%%%
\begin{thrm}
  Given a function \(f(x)\) such that \(f(x) = \sum_{n=0}^\infty a_n
  x^n\) on some interval \(I\), then
  \begin{enumerate}
  \item \(\frac{d}{dx} f(x) = \)
  \item \(\int f(x)\ dx = \)
  \end{enumerate}
\end{thrm}
\begin{ex}
  Find the Maclaurin series of the following series using the theorem above.   \begin{enumerate}
  \item \(\frac{1}{(1-2x)^2}\). (Hint: use what
  you know from challenges \(5\) and \(7\) on the ``11.1+11.6''
  notesheet.) \\
  \vspace{2in}
  \item \(\ln(1+x^2)\) from the power series for \(\frac{x}{1+x^2}\).
  \end{enumerate}
\end{ex}
\begin{ex}
  Sometimes you can ``recenter'' a Taylor series.
  \begin{enumerate}
  \item Find the Taylor series of \(e^x\) at \(x=1\) using the
    Maclaurin series \(e^x = \sum_{n=0}^\infty \frac{x^n}{n!}\)
    \vspace{2in}
  \item Find the Taylor series of \(\ln(2+3x)\) at \(x=10\)
    \vspace{2in}
  \item Find the Taylor series of \(\frac{x+5}{2+3x}\) at \(x=-5\)
    \vspace{2in}
  \item Find the Taylor series of \(\frac{2}{x}\) at \(x=2\)
  \end{enumerate}
\end{ex}
%%%%%%%%%%%%%%%%%%%%%%%%%%%%%%%%%%%%%%%%%%%%%%%%%%
\end{document}
