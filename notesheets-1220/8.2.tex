\documentclass[12pt, a4paper]{article}
\usepackage{../notesheets}
%%%%%%%%%%%%%%%%%%%%%%%%%%%%%%%%%%%%%%%%%%%%%%%%%%
\author{Math 1220}
\title{Notesheet. Section 8.2: Partial Derivatives}
\date{}

\begin{document}
\maketitle
\nameline
%%%%%%%%%%%%%%%%%%%%%%%%%%%%%%%%%%%%%%%%%%%%%%%%%%
\vspace{-0.3in}
Remember that the derivative of \(f(x)\) at \(x=a\) is the ``rate of
change of \(f(x)\) at \(x=a\)'' and is defined by \[
  f'(a) := \lim_{h \to 0} \frac{f(a+h)-f(a)}{h}.
\]
With multivariable functions, ``rate
of change'' is ambiguous if we do not define which input we are
changing.
\begin{defi}
  Let \(f(x,y)\) be a function in two variables. We define the
  \de{partial derivatives} as
  \begin{enumerate}
  \item \(f_x(x,y) = \frac{\partial}{\partial x}f(x,y) = \)
  \item \(f_y(x,y) = \frac{\partial}{\partial y}f(x,y) = \)
  \end{enumerate}
  That is, \(f_x(x,y)\) is the rate of change of \(f(x,y)\) if \(x\)
  is varied and \(y\) is fixed and the opposite for \(f_y(x,y)\).
\end{defi}
\begin{ex}
  Let \(f(x,y) = x^2+xy+y^2\). Compute \(f_x\) and \(f_y\). Let
  \(g(x,y) = e^{x^2}\sin(y)\). Compute \(g_x\) and \(g_y\).
\end{ex}
\begin{rmk}
  \begin{enumerate}
  \item   We compute partial derivatives, say \(\frac{\partial}{\partial
    x}f(x,y)\), by letting the variable we are differentiating vary
  and fixing all the others, so in this case, letting \(x\) vary and
  pretending \(y\) is constant.
  \item The chain rule, product rule, and the quotient rule all still
    apply for partial derivatives.
  \end{enumerate}
\end{rmk}
\begin{ex}
  Evaluate \(f_x\) and \(f_y\) for the following functions
  \begin{enumerate}
  \item \(f(x,y) = \ln(7+xy^2)\)
    \vspace{1in}
  \item \(f(x,y) = \frac{x-y}{x+y}\)
  \end{enumerate}
\end{ex}
\vspace{-1in}
\begin{thrm}
  Just like \(f'(a) = 0 \iff \) the tangent line of the graph of
  \(f(x)\) is horizontal, the tangent plane of \(z=f(x,y)\) at
  \((a,b)\) is horizontal if and only if 
\end{thrm}
\begin{defi}
  \(\left(\frac{\partial}{\partial x}\right)^2 f(x,y) =
  \frac{\partial^2}{\partial x^2} f(x,y) = f_{xx}(x,y)\) are
  all different notation for the same second partial derivative. Similarly,
  \(\frac{\partial}{\partial x}\left( \frac{\partial}{\partial y}
    f(x,y) \right) = \frac{\partial^2}{\partial x \partial y} f(x,y) =
  f_{yx}(x,y)\) are all different notations for the same partial derivative.
\end{defi}
\begin{ex}
  Let \(f(x,y) = x^3 + x^2 y + y^2\). Compute \(f_{xx}(x,y),
  f_{xy}(x,y), f_{yx}(x,y), \) and \(f_{yy}(x,y)\). Notice anything?
  Hint: Compute 
  \(f_x(x,y)\) and \(f_y(x,y)\) first. 
\end{ex}
\begin{thrm}
  If \(f_{xy}\) and \(f_{yx}\) are continuous, then 
\end{thrm}
%%%%%%%%%%%%%%%%%%%%%%%%%%%%%%%%%%%%%%%%%%%%%%%%%% 
\end{document}
