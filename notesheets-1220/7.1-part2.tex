\documentclass[12pt, a4paper]{article}
\usepackage{../notesheets}
%%%%%%%%%%%%%%%%%%%%%%%%%%%%%%%%%%%%%%%%%%%%%%%%%%
\author{Math 1220}
\title{Notesheet. Section 7.1: 
  Integration by parts part II}
\date{}

\begin{document}
\maketitle
\nameline
%%%%%%%%%%%%%%%%%%%%%%%%%%%%%%%%%%%%%%%%%%%%%%%%%%
\begin{rmk}
  When picking \(u\) and \(dv\) in integration by parts, we want to
  choose so that
  \begin{enumerate}
  \item \(du\) is simpler than \(u\).
  \item \(dv\) is easy to integrate.
  \end{enumerate}
\end{rmk}
\begin{ex}
  Use integration by parts to solve the following integrals
  \begin{enumerate}
  \item \(\int \ln x \dx\)
    \vspace{1in}
  \item \(\int x \ln x \dx\)
    \vspace{1in}
  \item \(\int x (x+4)^{-2} \dx\)
    \vspace{1in}
  \item \(\int e^x \sin x \dx\). (Hint: use integration by parts twice!)
  \end{enumerate}
\end{ex}
\begin{ex}
  Find the average value of \(f(x) = x^2 \ln x\) on the interval \([1,3]\).
\end{ex}
\begin{ex}
  Evaluate \(\int_1^3 x f''(x) \dx\) if \(f(1) = 1, f'(1)=0, f(3)=2,
  f'(3)=1\).
\end{ex}
\begin{ex}
  Sometimes integrals get really tricky, requiring you to use both
  \(u\)-substition and integration by parts! Solve the following integrals.
  \begin{enumerate}
  \item \(\int_{\sqrt{\pi/2}}^{\sqrt{\pi}} \theta^3 \cos(\theta^2)\
    d\theta\)
    \vspace{1in}
  \item \(\int \cos \sqrt{x} \dx\)
    \vspace{1in}
  \item \(\int \sin(\ln(x)) \dx\)
  \end{enumerate}
\end{ex}
%%%%%%%%%%%%%%%%%%%%%%%%%%%%%%%%%%%%%%%%%%%%%%%%%% 
\end{document}
