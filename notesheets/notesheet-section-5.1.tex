\documentclass[12pt, a4paper]{article}
\usepackage{../notesheets}

%%%%%%%%%%%%%%%%%%%%%%%%%%%%%%%%%%%%%%%%%%%%%%%%%%
\author{Math 1210}
\title{Notesheet. Section 5.1: Exponential Functions}
\date{}

\begin{document}
\maketitle
\nameline
%%%%%%%%%%%%%%%%%%%%%%%%%%%%%%%%%%%%%%%%%%%%%%%%%%
\begin{defi}
  An \de{exponential function} $f$ with base $b$ and exponent $x$ is the function
  % b > 0 required by Tan
  % b != 1 required, since it would be a constant function otherwise
\end{defi}
\begin{thrm}
  If $a, b \in \R$, $a, b > 0$, and $x, y \in \R$, then
  \begin{itemize}
    \item $b^x \cdot b^y = $
    \item $\frac{b^x}{b^y} = $
    \item $(b^x)^y = $
    \item $(ab)^x = $
    \item $\left(\frac{a}{b}\right)^x = $
  \end{itemize}
\end{thrm}
\begin{ex}
  \mbox{}
  \begin{itemize}
    \item Can you simplify $(81 \cdot 16)^{-\frac{1}{4}}$?
    \item Can you simplify $\frac{30^{-2/3}}{30^{1/3}}$?
    \item Can you simplify $\left(\frac{5^{1/6}}{5^{1/12}}\right)^6$?

  \end{itemize}
\end{ex}
\begin{ex}
  Can you sketch the graph of the function $f(x) = \left(\frac{1}{3}\right)^x$?
\end{ex}
\vspace{-0.55in}
\begin{thrm}
  The exponential function $f(x) = b^x$, $b > 0$, $b \neq 1$, has the following properties:
  \begin{itemize}
    \item Its domain is
    \item Its range is
    \item Its graph always passes through the point
    \item It is continuous on
    \item If $b >1$, then it is increasing on \phantom{$(-\oo, \oo)$}.  If $b <1$, it is decreasing on \phantom{$(-\oo, \oo)$}.
  \end{itemize}
\end{thrm}
\vspace{-1.00in}
\begin{ex}
  Can you sketch the graph of the function $f(x) = \left(\frac{1}{3}\right)^{-x}$?
\end{ex}
\vspace{-0.90in}
\begin{defi}
  $e$ is a very magical number that is \emph{approximately} equal to $2.7182818$.  $e$ is \emph{exactly} equal to $\lim_{n \to \oo} \left( 1 + \frac{1}{n} \right)^n$.  $e$ is also the ONLY number $b$ (other than 0) such that $\frac{d}{dx}[b^x] = b^x$!!!
\end{defi}
\vspace{-0.90in}
\begin{ex}
  Can you sketch the graph of the function $f(x) = e^{x}$?
\end{ex}
%%%%%%%%%%%%%%%%%%%%%%%%%%%%%%%%%%%%%%%%%%%%%%%%%%
\end{document}
